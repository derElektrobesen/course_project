\section{Технологический раздел}

\subsection{Среда разработки}
Для написания драйвера фильтра клавиатуры была выбрана операционная система \gnulinux.
Ядро данной операционной системы защищено лицензией GNU General Public License, согласно
которой его можно копировать, модифицировать и распространять.
\begin{flushleft}
	\begin{tabular}{ll}
		Операционная система:&\gnulinux\\
		Версия ядра:&\linuxv\\
		Дистрибутив:&\archlinux
	\end{tabular}
\end{flushleft}

\subsection{Выбор языка программирования}
Как и все <<уважающие себя>> ядра \unix, ядро \linux\ написано на языке \src{C}.
Но ядро \linux\ написано не на числом \src{C} в стандарте \src{ANSI C}. Где возможно,
разработчики ядра используют различные расширения языка, которые доступны
с помощью средств компиляции \src{gcc} (\src{GNU Compiler Collection}, где содержится
компилятор \src{C}, используемый для компиляции ядра).
\newpar
Разработчики ядра используют как расширения языка \src{C ISO C99}\footnote{Стандарт
\src{ISO C99}~--- последняя основная версия редакции стандарта \src{ISO C}.}, так и
расширения \src{GNU C}\ \cite{Love}.
\newpar
Поэтому, для написания модуля ядра был использован язык \src{C}.

\subsection{Среда программирования}
Разработка программного кода драйвера осуществлялась в текстовом редакторе \src{Vim}.
Ныне это один из мощнейших текстовых редакторов с полной свободой настройки 
и автоматизации, и созданным благодаря этому расширениям и надстройкам.

\newpar
Для компиляции исходного кода модуля была использована утилита \src{make}.
Данная утилита используя \src{Makefile} автоматизирует процесс компиляции и сборки
модуля ядра.
\newpar

Для отладки модуля была использована дополнительная виртуальная операционная система,
которая по своим параметрам полностью соответствует основной ОС, а именно:
\begin{flushleft}
	\begin{tabular}{ll}
		Операционная система:&\gnulinux\\
		Версия ядра:&\linuxv\\
		Дистрибутив:&\archlinux
	\end{tabular}
\end{flushleft}

\newpar
Отладочные сообщения позволяла просматривать утилита \src{dmesg} в сочетании с 
утилитой \src{grep}.

\subsection{Установка и использование программного обеспечения}
Для корректной работы программного обеспечения, его рекомендуется устанавливать
на систему с теми же характеристиками, которые были описаны выше. Однако установка
может так же производиться на операционные системы \gnulinux\ использующие ядро старше версии 
2.6.33\footnote{Именно в этой версии ядра были произведены последние изменения необходимых
для работы драйвера файлов.}. 
Так же для корректной генерации карты симполов необходимы библиотеки Python версии 3.3.2 или старше.
\newpar
Для компиляции проекта, неоходимо скопировать все модули проекта в текущую папку и запустить
утилиту \src{make}, предварительно отредактировав нужным образом переменную \src{KERNEL\_DIR} и,
при необходимости отладки, добавить в переменную \src{DEFINES} строку <<\src{DEBUG}>>.
\newpar

После успешной компиляции и сборки проекта в текущей директории появится файл \src{dFilter.ko}.
Для загрузки модуля в память, используйте команду \src{insmod dFilter.ko}, а для выгрузки~--
\src{rmmod dFilter}. В случае отладочной сборки, сообщения можно просматривать с помощью команды 
\src{dmesg | grep dFilter}\footnote{Для Вашего дистрибутива считывание сообщений может осуществляться иным образом.
	Если у Вас не установлена утилита \src{dmesg}, попробуйте считать лог-файл вручную: 
	\src{cat /var/log/messages | grep dFitler}.}.
\newpar

Для отображения текущей карты клавиш, необходимо выполнить команду
\src{./key\_codes.py}\footnote{Если расположение исполняемого файла \src{python} отлично от 
\linuxpath{/usr/bin/python}, попробуйте явно указать иной путь в shebang-строке скрипта или же явно передать скрипт на
выполнение установленной версии \src{python}.} предварительно выставив с помощью 
команды \src{chmod +x ./key\_codes.py} флаг выполнения для
файла \src{key\_codes.py}.
\newpar

Для изменения карты клавиш, отредактируйте файл \src{key\_map.conf}. Файл должен быть построен по следующим правилам:\\
\key{Код реальной клавиши} $->$ \key{Маска клавиш через пробел}\\
Маска клавиш указана в начале файла в блоке комментариев.
Комментарий начинается с символа \key{\#}. В случае, если в маске указан символ, который не был изначально предусмотрен
как часть маски или ключа, драйвер работать не будет. Внести коррективы в набор используемых
символов можно в файле \src{key\_codes.py} в переменной \src{key\_map}. Эта переменная представляет собой
словарь вида\\ \key{Символ} : \key{Скан-код символа}.
\newpar

Изменения в вышеуказанные файлы вводятся исключительно на стах и риск пользователя.
\newpage
\subsection{Исходные коды}
\subsubsection*{Заголовочный модуль \src{main\_header.h}}
\lstinputlisting[language = C, caption = {}]{../src/main_header.h}
\newpage
\subsubsection*{Модуль \src{main.c}}
\lstinputlisting[language = C, caption = {}]{../src/main.c}
\newpage
\subsubsection*{Заголовочный модуль \src{keyboard.h}}
\lstinputlisting[language = C, caption = {}]{../src/keyboard.h}
\newpage
\subsubsection*{Модуль \src{keyboard.c}}
\lstinputlisting[language = C, caption = {}]{../src/keyboard.c}
\newpage
\subsubsection*{Генерируемый заголовочный модуль \src{key\_map.h}}
\lstinputlisting[language = C, caption = {}]{../src/key_map.h}
\newpage
\subsubsection*{\src{Makefile}}
\lstinputlisting[caption = {}]{../Makefile}
\newpage
\subsubsection*{Скрипт \src{key\_codes.py}}
\lstinputlisting[language = Python, caption = {}]{../key_codes.py}
\newpage
\subsubsection*{Конфигурационный файл \src{key\_map.conf}}
\lstinputlisting[caption = {}]{../key_map.conf}
\newpage
