\addcontentsline{toc}{section}{Введение}
\section*{Введение}

Клавиатура -- одно из самых важных переферийных устройств компьютера. Не смотря
на совершенствование в наше время таких устройств ввода как мыши, трекболы, тачпады,
тачскрины и многих других, клавиатура все же остается основным и самым удобным средством
ввода информации. 
\newpar
Очень часто возникает необходимость в переназначении клавиш на клавиатуре:
\begin{itemize}
	\item Стандартное расположение клавиш неудобно при работе
		с определенным программным обеспечением;
	\item При переходе с одного типа клавиатуры на другой трудно привыкнуть
		к новому расположению клавиш.
\end{itemize}
\newpar

Длительное использование клавиатуры часто приводит к западанию клавиш.
Для решения данной проблемы можно использовать дополнительные методы ввода,
например имитация цифровой клавиатуры телефона с помощью дополнительной цифровой клавиатуры 
компьютера.
\newpar

Под операционной системой \linux\ существуют стандартные средства и программы, 
которые частично выполняют поставленную задачу. 
Примером стандартных средств может служить задание собственной маски клавиш клавиатуры
в файле \linuxpath{/usr/share/kbd/keymaps/i386/qwerty/personal.map}\footnote{Действительно для %
дистрибутива \fulllinux} в любимом текстовом редакторе и задание полученой
кодировки в файле \linuxpath{/etc/vconsole.conf}\,\footnotemark[1].
Примером программ для переназначения функций клавиш клавиатуры может служить утилита
\linuxutil{xmodmap} из пакета \linuxutil{xorg-server-utils}\footnote{Подробнее, \linuxcommand{man xmodmap}}.\\
Средств для полной реализации поставленной задачи нет.
\newpage
